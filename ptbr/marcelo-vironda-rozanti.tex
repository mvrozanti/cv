%%%%%%%%%%%%%%%%%
% This is an sample CV template created using altacv.cls
% (v1.1.5, 1 December 2018) written by LianTze Lim (liantze@gmail.com). Now compiles with pdfLaTeX, XeLaTeX and LuaLaTeX.
%
%% It may be distributed and/or modified under the
%% conditions of the LaTeX Project Public License, either version 1.3
%% of this license or (at your option) any later version.
%% The latest version of this license is in
%%    http://www.latex-project.org/lppl.txt
%% and version 1.3 or later is part of all distributions of LaTeX
%% version 2003/12/01 or later.
%%%%%%%%%%%%%%%%

%% If you need to pass whatever options to xcolor
\PassOptionsToPackage{dvipsnames}{xcolor}

%% If you are using \orcid or academicons
%% icons, make sure you have the academicons
%% option here, and compile with XeLaTeX
%% or LuaLaTeX.
% \documentclass[10pt,a4paper,academicons]{altacv}

%% Use the "normalphoto" option if you want a normal photo instead of cropped to a circle
% \documentclass[10pt,a4paper,normalphoto]{altacv}

\documentclass[10pt,a4paper,ragged2e]{altacv}

%% AltaCV uses the fontawesome and academicon fonts
%% and packages.
%% See texdoc.net/pkg/fontawecome and http://texdoc.net/pkg/academicons for full list of symbols. You MUST compile with XeLaTeX or LuaLaTeX if you want to use academicons.

% Change the page layout if you need to
\geometry{left=0.5cm,right=9cm,marginparwidth=6.8cm,marginparsep=1cm,top=1cm,bottom=1.25cm}

% Change the font if you want to, depending on whether
% you're using pdflatex or xelatex/lualatex
\ifxetexorluatex
  % If using xelatex or lualatex:
  \setmainfont{Lato}
\else
  % If using pdflatex:
  \usepackage[utf8]{inputenc}
  \usepackage[T1]{fontenc}
  \usepackage[default]{lato}
\fi
\usepackage{hyperref}
\hypersetup{hidelinks}

% Change the colours if you want to
\definecolor{Mulberr}  {HTML}{72243D}
\definecolor{SlateGrey}{HTML}{9E2E2E}
\definecolor{LightGrey}{HTML}{000000}
\colorlet{heading}{black}
\colorlet{accent}{Mulberr}
\colorlet{emphasis}{SlateGrey}
\colorlet{body}{LightGrey}

% Change the bullets for itemize and rating marker
% for \cvskill if you want to
\renewcommand{\itemmarker}{{\small\textbullet}}
\renewcommand{\ratingmarker}{\faCircle}

%% sample.bib contains your publications
\addbibresource{marcelo-vironda-rozanti.bib}

\begin{document}
\name{Marcelo Vironda Rozanti}
\tagline{Desenvolvedor Backend}
% \photo{3.7cm}{moi}
\personalinfo{%
  % Not all of these are required!
  % You can add your own with \printinfo{symbol}{detail}
  \email{mvrozanti@hotmail.com}
  \location{Rua Piauí 163 São Paulo - SP}
  \homepage{mvrozanti.github.io} \\
  \linkedin{linkedin.com/in/mvrozanti}
  \github{github.com/mvrozanti}
  %% You MUST add the academicons option to \documentclass, then compile with LuaLaTeX or XeLaTeX, if you want to use \orcid or other academicons commands.
  % \orcid{orcid.org/0000-0000-0000-0000}
}

\begin{fullwidth}
\makecvheader
\end{fullwidth}
\cvsection[marcelo-vironda-rozanti-p1sidebar]{Experiência}

\cvevent{Estagiário}{\href{http://www.azesconsultoria.com.br}{AzEs Consultoria Empresarial}}{2016 -- 2018}{São Paulo}
Modelagem de expressões regulares para captação e batimento de informações em folhas de pagamento. Criação de aplicação desktop para conduzir a captação.
\\

\cvtag{\texttt{git}}
\cvtag{Java SE}
\cvtag{Java Swing}
\cvtag{Ant}
\cvtag{MSSQL}
\cvtag{Batch scripts}

\divider

\cvevent{Desenvolvedor}{\href{https://www.sys4bank.com.br}{Sys4Bank}}{03/2018 -- 01/2020}{São Paulo}
Desenvolvimento de soluções bancárias de back e front ends para setores Corporativo, Financeiro, de Crédito e de Conta. Primeiro contato com práticas de CI/CD e Spring Framework. Modelagem diária de banco de dados MSSQL.
\\

\cvtag{GNU Coreutils}
\cvtag{Java EE}
\cvtag{MSSQL}
\cvtag{Maven}
\cvtag{BPM}
\cvtag{Bash scripts}
\cvtag{Travis CI}

\divider

\cvevent{Desenvolvedor}{\href{https://www.rd.com.br}{Raia Drogasil}}{01/2020 -- 12/2020}{São Paulo}
Soluções de Operações de Venda com foco na conformidade com a Lei Geral de Proteção de Dados (LGPD). Atualização de software legado. Desenvolvimento de software para integração e interfaceamento com leitores biométricos e impressoras térmicas. Colaboração baseada em times.
\\

\cvtag{Java EE}
\cvtag{Java Applets}
\cvtag{DTAP}
\cvtag{TDD}
\cvtag{Oracle SQL Server}\\
\cvtag{JSF/Primefaces}
\cvtag{JBoss}
\cvtag{WebLogic}
\cvtag{Spring Framework}

\divider

\cvevent{Desenvolvedor Backend}{\href{https://www.grao.com.br}{Grão}}{12/2020 -- 09/2021}{São Paulo}
Responsável pela implementação do serviço de pagamento instantâneo Pix em Kotlin. Ambientes de homologação conteinerizados para testes e produção. Tarefas diárias: modelagem de entidades com foco em performance de escrita e acesso no banco de dados. Desenvolvimento de endpoints REST escaláveis. Testes de QA. Ajustes pontuais em front end que consome a API.
\\

\cvtag{Java EE}
\cvtag{Kotlin}
\cvtag{MySQL}
\cvtag{Lens}
\cvtag{GitLab Pipelines}\newline
\cvtag{Google Firebase}
\cvtag{Spring Framework}
\cvtag{Feign Client}
\cvtag{Graylog}\\
\cvtag{Newman/Postman}
\cvtag{RD Station}
\cvtag{Tomcat}
\cvtag{Svelte}

\divider

\cvevent{Desenvolvedor Backend}{\href{https://www.contaazul.com}{Conta Azul}}{09/2021 -- 08/2022}{São Paulo}

Desenvolvimento de soluções contábeis. Responsável pela implementação de um sistema para gerenciamento automático de obrigações fiscais periódicas e aperiódicas relacionadas ao SPED/eSocial, EFD-Reinf.
\\

\cvtag{Java 8, 11, 15, 16}
\cvtag{AWS Queues, Storage and Infrastructure}\\
\cvtag{Terraform/HCL}
\cvtag{PostgreSQL}
\cvtag{LogDNA}
\cvtag{Spring Framework}

\divider

\cvevent{Analista de Sustentação}{\href{https://www.actdigital.com}{act digital}}{03/2024 -- presente}{São Paulo}

Desenvolvimento, sustentação, implantação e monitoramento de softwares relacionados a dispositivos de POS (\textit{Point of sale}), as "maquininhas" de pagamento. Forte atuação em DevSecOps, esteira de CI/CD, e documentação ténica e processual.
\\

\cvtag{Splunk}
\cvtag{Dynatrace}
\cvtag{Spring Framework}
\cvtag{Oracle SQL Server}\\
\cvtag{Docker}
\cvtag{Fortify}
\cvtag{Sonar}
\cvtag{Mercurial}
\cvtag{CyberArk}

\clearpage
% \cvsection[page2sidebar]{Publications}

% \nocite{*}

% \printbibliography[heading=pubtype,title={\printinfo{\faBook}{Books}},type=book]

% \divider

% \printbibliography[heading=pubtype,title={\printinfo{\faFileTextO}{Journal Articles}},type=article]

% \divider

\printbibliography[heading=pubtype,title={\printinfo{\faGroup}{Conference Proceedings}},type=inproceedings]

%% If the NEXT page doesn't start with a \cvsection but you'd
%% still like to add a sidebar, then use this command on THIS
%% page to add it. The optional argument lets you pull up the
%% sidebar a bit so that it looks aligned with the top of the
%% main column.
% \addnextpagesidebar[-1ex]{page3sidebar}


\end{document}
