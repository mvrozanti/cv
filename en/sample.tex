%%%%%%%%%%%%%%%%%
% This is an sample CV template created using altacv.cls
% (v1.1.5, 1 December 2018) written by LianTze Lim (liantze@gmail.com). Now compiles with pdfLaTeX, XeLaTeX and LuaLaTeX.
%
%% It may be distributed and/or modified under the
%% conditions of the LaTeX Project Public License, either version 1.3
%% of this license or (at your option) any later version.
%% The latest version of this license is in
%%    http://www.latex-project.org/lppl.txt
%% and version 1.3 or later is part of all distributions of LaTeX
%% version 2003/12/01 or later.
%%%%%%%%%%%%%%%%

%% If you need to pass whatever options to xcolor
\PassOptionsToPackage{dvipsnames}{xcolor}

%% If you are using \orcid or academicons
%% icons, make sure you have the academicons
%% option here, and compile with XeLaTeX
%% or LuaLaTeX.
% \documentclass[10pt,a4paper,academicons]{altacv}

%% Use the "normalphoto" option if you want a normal photo instead of cropped to a circle
% \documentclass[10pt,a4paper,normalphoto]{altacv}

\documentclass[10pt,a4paper,ragged2e]{altacv}

%% AltaCV uses the fontawesome and academicon fonts
%% and packages.
%% See texdoc.net/pkg/fontawecome and http://texdoc.net/pkg/academicons for full list of symbols. You MUST compile with XeLaTeX or LuaLaTeX if you want to use academicons.

% Change the page layout if you need to
\geometry{left=1cm,right=9cm,marginparwidth=6.8cm,marginparsep=1.2cm,top=1.25cm,bottom=1.25cm}

% Change the font if you want to, depending on whether
% you're using pdflatex or xelatex/lualatex
\ifxetexorluatex
  % If using xelatex or lualatex:
  \setmainfont{Lato}
\else
  % If using pdflatex:
  \usepackage[utf8]{inputenc}
  \usepackage[T1]{fontenc}
  \usepackage[default]{lato}
\fi
\usepackage{hyperref}

% Change the colours if you want to
\definecolor{Mulberr}  {HTML}{72243D}
\definecolor{SlateGrey}{HTML}{9E2E2E}
\definecolor{LightGrey}{HTML}{666666}
\colorlet{heading}{black}
\colorlet{accent}{Mulberr}
\colorlet{emphasis}{SlateGrey}
\colorlet{body}{LightGrey}

% Change the bullets for itemize and rating marker
% for \cvskill if you want to
\renewcommand{\itemmarker}{{\small\textbullet}}
\renewcommand{\ratingmarker}{\faCircle}

%% sample.bib contains your publications
\addbibresource{sample.bib}

\begin{document}
\name{Marcelo Vironda Rozanti}
\tagline{Backend Developer}
\photo{2.8cm}{moi}
\personalinfo{%
  % Not all of these are required!
  % You can add your own with \printinfo{symbol}{detail}
  \email{mvrozanti@hotmail.com}
  \phone{(011) 97100-1145}
  \mailaddress{Rua Dr. Veiga Filho}
  \location{São Paulo, Brasil}
  \homepage{mvrozanti.github.io}
  % \twitter{@twitterhandle}
  \linkedin{linkedin.com/in/mvrozanti}
  \github{github.com/mvrozanti}
  %% You MUST add the academicons option to \documentclass, then compile with LuaLaTeX or XeLaTeX, if you want to use \orcid or other academicons commands.
  % \orcid{orcid.org/0000-0000-0000-0000}
}

%% Make the header extend all the way to the right, if you want.
\begin{fullwidth}
\makecvheader
\end{fullwidth}

%% Depending on your tastes, you may want to make fonts of itemize environments slightly smaller
% \AtBeginEnvironment{itemize}{\small}

%% Provide the file name containing the sidebar contents as an optional parameter to \cvsection.
%% You can always just use \marginpar{...} if you do
%% not need to align the top of the contents to any
%% \cvsection title in the "main" bar.
\cvsection[sample-p1sidebar]{Experience}

\cvevent{Estagiário}{\href{http://www.azesconsultoria.com.br}{AzEs Consultoria Empresarial}}{03 2016 -- 2018}{São Paulo}
Payroll document processor development in Java. First exposure to Agile Methodolgy MVC and OOP design patterns.

\divider

\cvevent{Desenvolvedor}{\href{https://www.sys4bank.com.br}{Sys4Bank}}{03 2019 -- Ongoing}{São Paulo}
Bank infrastucture development. Technologies employed: Git, SpringBoot, Containerization, CI/CD practices and DevOps philosophy. Focus on automation and pipeline implementation. 

\cvsection{Projects}
\cvevent{\href{https://github.com/mvrozanti/RAT-via-Telegram/}{RAT-via-Telegram} {\faStar}310+}{Windows Remote Administration Tool via Telegram}{04 2017 -- Ongoing}{}
% \begin{itemize}
% \item Details
% \end{itemize}

\divider

\cvevent{ \href{https://github.com/mvrozanti/yawc}{yawc}}{{\faWhatsapp} Yet Another WhatsApp Client}{10 2018 -- 04 2019}{}
% Yet Another WhatsApp Client 
% kek

\divider

\cvevent{\href{https://github.com/mvrozanti/dotty}{dotty}}{{\faSuitcase} Script for syncing and managing versions of your dotfiles.}{09 2018 -- Ongoing}{}

\divider

\cvevent{\href{https://github.com/mvrozanti/colp}{colp}}{{\faPaintBrush} Command-line color processor}{04 2019 -- Ongoing}{}

\medskip

\cvsection{A Day of My Life}

% Adapted from @Jake's answer from http://tex.stackexchange.com/a/82729/226
% \wheelchart{outer radius}{inner radius}{
% comma-separated list of value/text width/color/detail}
\wheelchart{1.5cm}{0.5cm}{%
  7/8em/accent!10/{Sleep,\\beautiful sleep},
  4/8em/accent!60/College,
  4/8em/accent!/Daytime job,
  4/8em/accent!40/Personal projects,
  3/10em/accent!30/Relaxation,
  2/6em/accent!20/Spending time with family
}

\clearpage
% \cvsection[page2sidebar]{Publications}

% \nocite{*}

% \printbibliography[heading=pubtype,title={\printinfo{\faBook}{Books}},type=book]

% \divider

% \printbibliography[heading=pubtype,title={\printinfo{\faFileTextO}{Journal Articles}},type=article]

% \divider

\printbibliography[heading=pubtype,title={\printinfo{\faGroup}{Conference Proceedings}},type=inproceedings]

%% If the NEXT page doesn't start with a \cvsection but you'd
%% still like to add a sidebar, then use this command on THIS
%% page to add it. The optional argument lets you pull up the
%% sidebar a bit so that it looks aligned with the top of the
%% main column.
% \addnextpagesidebar[-1ex]{page3sidebar}


\end{document}
