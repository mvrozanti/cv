%%%%%%%%%%%%%%%%%
% This is an sample CV template created using altacv.cls
% (v1.1.5, 1 December 2018) written by LianTze Lim (liantze@gmail.com). Now compiles with pdfLaTeX, XeLaTeX and LuaLaTeX.
%
%% It may be distributed and/or modified under the
%% conditions of the LaTeX Project Public License, either version 1.3
%% of this license or (at your option) any later version.
%% The latest version of this license is in
%%    http://www.latex-project.org/lppl.txt
%% and version 1.3 or later is part of all distributions of LaTeX
%% version 2003/12/01 or later.
%%%%%%%%%%%%%%%%

%% If you need to pass whatever options to xcolor
\PassOptionsToPackage{dvipsnames}{xcolor}

%% If you are using \orcid or academicons
%% icons, make sure you have the academicons
%% option here, and compile with XeLaTeX
%% or LuaLaTeX.
% \documentclass[10pt,a4paper,academicons]{altacv}

%% Use the "normalphoto" option if you want a normal photo instead of cropped to a circle
% \documentclass[10pt,a4paper,normalphoto]{altacv}

\documentclass[10pt,a4paper,ragged2e]{altacv}

%% AltaCV uses the fontawesome and academicon fonts
%% and packages.
%% See texdoc.net/pkg/fontawecome and http://texdoc.net/pkg/academicons for full list of symbols. You MUST compile with XeLaTeX or LuaLaTeX if you want to use academicons.

% Change the page layout if you need to
\geometry{left=0.5cm,right=9cm,marginparwidth=6.8cm,marginparsep=1cm,top=1.25cm,bottom=1.25cm}

% Change the font if you want to, depending on whether
% you're using pdflatex or xelatex/lualatex
\ifxetexorluatex
  % If using xelatex or lualatex:
  \setmainfont{Lato}
\else
  % If using pdflatex:
  \usepackage[utf8]{inputenc}
  \usepackage[T1]{fontenc}
  \usepackage[default]{lato}
\fi
\usepackage{hyperref}
\hypersetup{hidelinks}

% Change the colours if you want to
\definecolor{Mulberr}  {HTML}{72243D}
\definecolor{SlateGrey}{HTML}{9E2E2E}
\definecolor{LightGrey}{HTML}{666666}
\colorlet{heading}{black}
\colorlet{accent}{Mulberr}
\colorlet{emphasis}{SlateGrey}
\colorlet{body}{LightGrey}

% Change the bullets for itemize and rating marker
% for \cvskill if you want to
\renewcommand{\itemmarker}{{\small\textbullet}}
\renewcommand{\ratingmarker}{\faCircle}

%% sample.bib contains your publications
\addbibresource{sample.bib}

\begin{document}
\name{Marcelo Vironda Rozanti}
\tagline{Backend Developer}
\photo{4.5cm}{moi}
\personalinfo{%
  % Not all of these are required!
  % You can add your own with \printinfo{symbol}{detail}
  \email{mvrozanti@hotmail.com}
  \location{Rua Piauí 163, São Paulo - SP, Brasil}
  \homepage{mvrozanti.github.io}
  % \twitter{@twitterhandle}
  \hspace{0.7cm}\linkedin{linkedin.com/in/mvrozanti}
  \github{github.com/mvrozanti}
  %% You MUST add the academicons option to \documentclass, then compile with LuaLaTeX or XeLaTeX, if you want to use \orcid or other academicons commands.
  % \orcid{orcid.org/0000-0000-0000-0000}
}

%% Make the header extend all the way to the right, if you want.
\begin{fullwidth}
\makecvheader
\end{fullwidth}

%% Depending on your tastes, you may want to make fonts of itemize environments slightly smaller
% \AtBeginEnvironment{itemize}{\small}

%% Provide the file name containing the sidebar contents as an optional parameter to \cvsection.
%% You can always just use \marginpar{...} if you do
%% not need to align the top of the contents to any
%% \cvsection title in the "main" bar.
\cvsection[sample-p1sidebar]{Experience}

\cvevent{Intern}{\href{http://www.azesconsultoria.com.br}{AzEs Consultoria Empresarial}}{2016 -- 2018}{São Paulo}
Payroll document processor development in Java. First exposure to Agile Methodolgy MVC and OOP design patterns.
\\

\cvtag{\texttt{git}}
\cvtag{Java SE}
\cvtag{Java Swing}
\cvtag{Ant}
\cvtag{MSSQL}
\cvtag{Batch scripts}

\divider

\cvevent{Developer}{\href{https://www.sys4bank.com.br}{Sys4Bank}}{Mar 2018 -- Jan 2020}{São Paulo}
Bank infrastucture development. Technologies used Git, SpringBoot, Containerization, CI/CD practices and DevOps philosophy. Focus on automation and pipeline implementation. Daily MSSQL Database modeling/usage.
\\

\cvtag{GNU Coreutils}
\cvtag{Java EE}
\cvtag{MSSQL}
\cvtag{Maven}
\cvtag{BPM}
\cvtag{Bash scripts}
\cvtag{Travis CI}

\divider

\cvevent{Developer}{\href{https://www.rd.com.br}{Raia Drogasil}}{Jan 2020 -- Dec 2020}{São Paulo}
Selling Operations solutions, focused on compliance with brazilian "LGPD" data-protection law. Technologies used: Java Applets,\\ JSF/Primefaces, Hardware-specific software development including biometric readers and thermal printers. First contact with DTAP phased approach. Strong team-based production. Daily Oracle Database SQL modeling/usage.
\\

\cvtag{Java EE}
\cvtag{Java Applets}
\cvtag{DTAP}
\cvtag{TDD}
\cvtag{Oracle SQL Server}\\
\cvtag{JSF/Primefaces}
\cvtag{JBoss}
\cvtag{WebLogic}
\cvtag{Spring Framework}

\divider

\cvevent{Backend Developer}{\href{https://www.grao.com.br}{Grão}}{Dec 2020 -- Sep 2021}{São Paulo}
Implementing the backend for supporting the brazilian instant payment system Pix in Kotlin. Container-based environments for testing and production. Daily tasks: performance-focused database entity modeling and MVC-based, scalable REST API design and QA testing. 
\\

\cvtag{Java EE}
\cvtag{Kotlin}
\cvtag{MySQL}
\cvtag{Lens}
\cvtag{GitLab Pipelines}\newline
\cvtag{Google Firebase}
\cvtag{Spring Framework}
\cvtag{Feign Client}
\cvtag{Graylog}\\
\cvtag{Newman/Postman}
\cvtag{RD Station}
\cvtag{Tomcat}
\cvtag{Svelte}

\divider

\cvevent{Backend Developer}{\href{https://www.contaazul.com}{Conta Azul}}{Sep 2021 -- Aug 2022}{São Paulo}

Accounting solutions. Implemented a system to automatically manage periodic and aperiodic tax obligations related to brazilian SPED/eSocial government modules.
\\

\cvtag{Java 8, 11, 15, 16}
\cvtag{AWS Queues, Storage and Infrastructure}\\
\cvtag{Terraform/HCL}
\cvtag{PostgreSQL}
\cvtag{LogDNA}
\cvtag{Spring Framework}

%
% \cvsection{A Day of My Life}
%
% \wheelchart{1.5cm}{0.5cm}{%
%  8/8em/accent!10/{Sleep,\\beautiful sleep},
%  %/8em/accent!60/College,
%  8/8em/accent!/Daytime job,
%  3/8em/accent!40/Personal projects,
%  3/10em/accent!30/Relaxation,
%  2/6em/accent!20/Spending time with family
% }

\clearpage
% \cvsection[page2sidebar]{Publications}

% \nocite{*}

% \printbibliography[heading=pubtype,title={\printinfo{\faBook}{Books}},type=book]

% \divider

% \printbibliography[heading=pubtype,title={\printinfo{\faFileTextO}{Journal Articles}},type=article]

% \divider

\printbibliography[heading=pubtype,title={\printinfo{\faGroup}{Conference Proceedings}},type=inproceedings]

%% If the NEXT page doesn't start with a \cvsection but you'd
%% still like to add a sidebar, then use this command on THIS
%% page to add it. The optional argument lets you pull up the
%% sidebar a bit so that it looks aligned with the top of the
%% main column.
% \addnextpagesidebar[-1ex]{page3sidebar}


\end{document}
